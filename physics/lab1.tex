\documentclass[12pt, letterpaper]{article}
\usepackage{makecell}
\usepackage[left=2cm,right=2cm,
    top=0.8cm,bottom=1.5cm,bindingoffset=0cm]{geometry}
\usepackage{amsmath,amsthm,amssymb}
\usepackage[utf8]{inputenc}
\usepackage[T1]{fontenc}
\usepackage[english,russian]{babel}
\usepackage{pgfplots}
\pgfplotsset{height=13cm,width=13cm,compat=1.9}
\usepackage{graphicx}

\title{Лабораторная работа №1}
\author{Выполнили: Политучая Инна и Салдина Мария}
\date{}

\begin{document}

\maketitle
\textbf{Наименование работы:} Исследование колебаний пружинного маятника

\textbf{Цель работы:} изучение гармонических колебаний, измерение коэффициента жёсткости пружины и логарифмического декремента затухания пружинного маятника.

\textbf{Принадлежности:} установка, набор пружин, набор гирь, секундомер.

\textbf{Рабочая формула:}  Жёсткость пружины статическим методом рассчитывается по формуле закона Гука:
\[k = \frac{mg}{|x - x_0|},\eqno(1)\]
где \(k\) - жёсткость пружины, в г/с\(^2\),

\(m\) - масса пружины, в г,

\(g\) - ускорение свободного падения, \(g=980,665\) см/с\(^2\)

\(x\) - начальная длина пружины, в см,

\(x_0\) - длина растянутой пружины, в см.

Жёсткость пружины динамическим методом определяется по формуле:
\[k = 4\pi^2N^2\frac{m}{t^2},\eqno(2)\]
где \(N\) - количество последовательных колебаний маятника за время \(t\),

\(t\) - время N последовательных колебаний.

Декремент затухания можно рассчитать по формуле:
\[\theta = \frac{T_1}{t}\ln \frac{A_0}{A_t},\eqno(3)\]
где \(\theta\) - логарифмический декремент затухания,

\(T_1\) - условный период колебаний,

\(t=NT_1\) - момент времени, в с,

\(A_0\) - первоначальная амплитуда колебаний, в м,

\(A_t\) - амплитуда в момент времени \(t\), в м.

\begin{center}
\Large{\textbf{\newline\newline\newline\newline\newline\newline\newline\newline\newline Статический метод}}\
\end{center}


\begin{table}[h]
\caption{\label{tab:bolts} Статический метод}
\begin{center}
 \begin{tabular}{|c|c|c|c|c|c|c|c|c|c|c|} 
 \hline
\thead{\\Номер\\опыта} & \thead{\\Номер\\пружины} & \(x_0\), см & \(m\), г & \(x\), см & \(x - x_0\), см & \(k\), г/\( c^2\) & \(\overline{k}\), г/\( c^2\) & \(|\Delta k|\), г/с\(^2\) & \(|\overline{\Delta k}|\), г/с\(^2\) \\ [0.5ex] 
 \hline
 1 & 1 & 23,8 & 54 & 26,8 & 3 & 17652 & 16807 & 845 & 458,75 \\ 
 \hline
 2 & 2 & 22,4 & 54 & 25,8 & 3,4 & 15575 & 15292 & 283 & 222,5 \\
 \hline
 3 & 3 & 20,2 & 54 & 22,8 & 2,6 & 20368 & 22185 & 1817 & 908,25 \\
 \hline
 4 & 1 & 23,8 & 105 & 29,9 & 6,1 & 16880 & 16807 & 73 & 458,75 \\ 
 \hline
 5 & 2 & 22,4 & 105 & 29,3 & 6,9 & 14923 & 15292 & 369 & 222,5 \\
 \hline
 6 & 3 & 20,2 & 105 & 24,7 & 4,5 & 22882 & 22185 & 697 & 908,25 \\
 \hline
 7 & 1 & 23,8 & 156 & 33,2 & 9,4 & 16275 & 16807 & 532 & 458,75 \\ 
 \hline
 8 & 2 & 22,4 & 156 & 32,3 & 9,9 & 15453 & 15292 & 161 & 222,5 \\
 \hline
 9 & 3 & 20,2 & 156 & 27 & 6,8 & 22498 & 22185 & 313 & 908,25 \\
 \hline
 10 & 1 & 23,8 & 211 & 36,4 & 12,6 & 16422 & 16807 & 385 & 458,75 \\ 
 \hline
 11 & 2 & 22,4 & 211 & 36 & 13,6 & 15215 & 15292 & 77 & 222,5 \\
 \hline
 12 & 3 & 20,2 & 211 & 29,2 & 9 & 22991 & 22185 & 806 & 908,25 \\
 \hline
\end{tabular}
\end{center}
\end{table}

\begin{center}
\begin{tikzpicture}
\begin{axis}[
title = График зависимости \(|x - x_0|\) от \(m\),
legend pos = north west,
xlabel = \(m\) г, ylabel = \(|x - x_0|\) см,
xmin = 0, xmax = 220,
ymin = 0, ymax = 20,
grid=major
]
\legend{
Первая пружина,
Вторая пружина,
Третья пружина,
};

\addplot coordinates {
	(54,3) (105,6.1) (156,9.4) (211,12.6)
};
\addplot coordinates {
	(54,3.4) (105,6.9) (156,9.9) (211,13.6)
};
\addplot coordinates {
	(54,2.6) (105,4.5) (156,6.8) (211,9)
};
\addplot[dashed] coordinates {
	(54,3) (211,12.6)
};
\addplot[dashed] coordinates {
	(54,3.4) (211,13.6)
};
\addplot[dashed] coordinates {
	(54,2.6) (211,9)
};
\end{axis}
\end{tikzpicture}
\end{center}

Область выполнения закона Гука: 1 - [54, 120], 2 - закон Гука не выполняется, 3 - [140, 211]

\begin{table}[h]
\caption{\label{tab:bolts} Статический метод (определение коэффициента жёсткости по графику)}
\begin{center}
 \begin{tabular}{|c|c|c|c|c|} 
 \hline
\thead{\\Номер\\пружины} & \(\alpha\) & \(\tg \alpha\) & \(k\), г/\(c^2\) & Отклонение от результата вычислений, \% \\ [0.5ex] 
 \hline
1 & 34 & 0.6745 & 15993 & 4,8 \\
\hline
2 & 37 & 0.7536 & 14314,4 & 6,4 \\
\hline
3 & 24 & 0,4452 & 24230,3 & 9,2 \\
\hline
\end{tabular}
\end{center}
\end{table}

\begin{center}
\Large{\textbf{Динамический метод}}\
\end{center}

\begin{table}[h]
\caption{\label{tab:bolts} Динамический метод}
\begin{center}
 \begin{tabular}{|c|c|c|c|c|c|c|c|c|c|c|c|c|} 
 \hline
\thead{\\Номер\\опыта} & \thead{\\Номер\\пружины} & \(t\), с & \(m\), г & \(N\) & \(t^2\), с & \(N^2\) & \thead{\(\frac{4\pi^2N^2}{t^2}\),\\\( c^{-2}\)} & \(k\), г/\( c^2\) & \(\overline{k}\), г/с\(^2\) & \thead{\\\(|\Delta k|\),\\г/с\(^2\)} & \thead{\\\(|\overline{\Delta k}|\),\\г/с\(^2\)} \\ [0.5ex] 
 \hline
 1 & 1 & 11,8 & 80 & 25 & 139,24 & 625 & 177,025 & 14162 & 15390 & 1228 & 614 \\ 
 \hline
 2 & 2 & 11,5 & 80 & 25 & 132,25 & 625 & 186,382 & 14910,56 & 15233 & 322 & 453 \\
 \hline
 3 & 3 & 9,9 & 80 & 25 & 98,01 & 625 & 251,495 & 20119,6 & 21420 & 1300 & 981 \\
 \hline
 4 & 1 & 14,3 & 131 & 25 & 204,49 & 625 & 120,539 & 15790,609 & 15390 & 400 & 614 \\ 
 \hline
 5 & 2 & 14,5 & 131 & 25 & 210,25 & 625 & 117,237 & 15358,047 & 15233 & 125 & 453 \\
 \hline
 6 & 3 & 12,2 & 131 & 25 & 148,84 & 625 & 165,607 & 21694,517 & 21420 & 275 & 981 \\
 \hline
 7 & 1 & 16,7 & 182 & 25 & 278,89 & 625 & 88,383 & 16085,706 & 15390 & 696 & 614 \\ 
 \hline
 8 & 2 & 17,5 & 182 & 25 & 306,25 & 625 & 80,487 & 14648,634 & 15233 & 584 & 453 \\
 \hline
 9 & 3 & 14,7 & 182 & 25 & 216,09 & 625 & 114,068 & 20760,376 & 21420 & 660 & 981 \\
 \hline
 10 & 1 & 19,4 & 237 & 25 & 376,36 & 625 & 65,493 & 15521,841 & 15390 & 132 & 614 \\ 
 \hline
 11 & 2 & 19,1 & 237 & 25 & 364,81 & 625 & 67,567 & 16013,379 & 15233 & 780 & 453 \\
 \hline
 12 & 3 & 15,9 & 237 & 25 & 252,81 & 625 & 97,5 & 23107,5 & 21420 & 1687 & 981 \\
 \hline
\end{tabular}
\end{center}
\end{table}

\begin{tikzpicture}
\begin{axis}[
title = График зависимости \(\frac{4\pi^2N^2}{t^2}\) от \(m\),
legend pos = south west,
xlabel = \(m\) г, ylabel = \(\frac{4\pi^2N^2}{t^2}\) \( c^{-2}\),
xmin = 0, xmax = 260,
ymin = 0, ymax = 260,
grid=major
]
\legend{
Первая пружина,
Вторая пружина,
Третья пружина,
};

\addplot coordinates {
	(80,177.025) (131,120.539) (182,88.383) (237,65.493)
};
\addplot coordinates {
	(80,186.382) (131,117.237) (182,80.487) (237,67.567)
};
\addplot
 coordinates {
	(80,251.495) (131,165.607) (182,114.068) (237,97.5)
};
\end{axis}
\end{tikzpicture}


\begin{table}[h]
\caption{\label{tab:bolts} Сравнение результатов}
\begin{center}
\begin{tabular}{|c|c|c|c|c|c|c|}
\hline
& \multicolumn{6}{c|}{Методы} \\
\cline{2-7}
& \multicolumn{3}{c|}{статический} & \multicolumn{3}{c|}{динамический} \\
\cline{2-7}
\raisebox{3ex}[0cm][0cm]{Номер пружины}
& \(\overline{k}\), г/с\(^2\) & \(\overline{|\Delta k}|\), г/с\(^2\) & \thead{Погрешность\\относительно\\дин.} & \(\overline{k}\), г/с\(^2\) & \(\overline{|\Delta k}|\), г/с\(^2\) & \thead{Погрешность\\относительно\\стат.} \\
\hline
1 & 16807 & 458,75 & 8,43 \% & 15390 & 614 & 9,2 \% \\
\hline
2 & 15292 & 222,5 & 0,39 \% & 15233 & 453 & 0,39 \% \\
\hline
3 & 22185 & 908,25 & 3,45 \% & 21420 & 981 & 3,57 \% \\
\hline
\end{tabular}
\end{center}
\end{table}

\begin{center}
\Large{\textbf{\newline\newline\newline\newline\newline\newline\newline Определение логарифмического декремента}}\
\end{center}

\begin{table}[h]
\caption{\label{tab:bolts} Логарифмический декремент}
\begin{center}
 \begin{tabular}{|c|c|c|c|c|c|c|c|c|} 
 \hline
Номер опыта & Номер пружины & \(A_0\), см & \(t_1\), с & \(N\) & \(T_1\), с & \(t\), с & \(A_t\), см & \(\Theta\) \\ [0.5ex] 
 \hline
 1 & 1 & 4 & 16,1 & 25 & 0,644 & 12 & 2 & 0,037199 \\ 
 \hline
 2 & 1 & 3 & 15,6 & 25 & 0,624 & 11,3 & 1,5 & 0,0382764 \\
 \hline
 3 & 1 & 5 & 16,4 & 25 & 0,656 & 13,67 & 2,5 & 0,033263 \\
 \hline
 4 & 2 & 4 & 16,4 & 25 & 0,656 & 13,8 & 2 & 0,0329496 \\
 \hline
 5 & 2 & 3 & 15,85 & 25 & 0,634 & 13,52 & 1,5 & 0,03250409 \\ 
  \hline
 6 & 2 & 5 & 17,2 & 25 & 0,656 & 14,25 & 2,5 & 0,033263 \\
 \hline
 7 & 3 & 4 & 15,2 & 25 & 0,4 & 15,42 & 2 & 0,027330317 \\
 \hline
 8 & 3 & 3 & 14,5 & 25 & 0,58 & 15 & 1,5 & 0,0268017 \\
  \hline
 9 & 3 & 5 & 15,8 & 25 & 0,632 & 16,1 & 2,5 & 0,02720926 \\
 \hline
\end{tabular}
\end{center}
\end{table}

\textbf{Вывод:}\ в ходе работы были измерены коэффициенты жёсткости (статическим и динамическим методами) и логарифмический декремент для трёх пружин.

\begin{center}
\Large{\textbf{\newline Контрольные вопросы}}\
\end{center}

1. \textit{Какие колебания называют гармоническими?}

Кинематическое определение: колебательный процесс, при котором отклонение колеблющейся величины происходит по закону синуса или косинуса.

Динамическое определение: колебательный процесс, для которого возвращающая сила F прямо пропорциональна отклонению x от положения равновесия, то есть движение происходит под действием упругой силы \(F=-kx\).\newline

2. \textit{При каких условиях обеспечиваются гармонические колебания? Какими параметрами они характеризуются?}

Условие: отсутствие сил трения, приводящих к затуханию колебаний (нет потерь энергии).

Уравнение движения:
\[m\ddot{x}=-kx+P, \eqno(4)\]
В системе координат, начало которой совпадает с положением равновесия пружины:
\[m\ddot{z}=-kz\]
откуда
\[z=A\sin(\omega t+\phi),\]
где \(\omega =\sqrt{\frac{k}{m}}\), т.е. период равен \(T=2\pi\sqrt{\frac{m}{k}}\)

Параметры:

- амплитуда

- частота и период

- фаза колебаний.\newline

3. \textit{От каких величин зависит период колебаний пружинного маятника?}

От массы грузика и коэффициента жёсткости пружины.\newline

4. \textit{В чём состоят статический и динамический методы измерения жёсткости пружины?}

Статический: измерить массу грузика и удлинение пружины при его подвешивании; по формуле закона Гука рассчитать коэффициент жёсткости.

Динамический: измерить массу грузика и время, за которое маятник совершает некоторое целое число колебаний вдоль вертикальной оси; по формуле (2) рассчитать значение коэффициента жёсткости.\newline

5. \textit{В чём состоит цель измерения коэффициента жёсткости пружины двумя методами?}

Если значение, полученное динамическим методом, совпадает (в пределах погрешности) со значением, полученным статическим методом, это указывает на гармонический характер колебаний. (т.к. формула (2), используемая в динамическом методе, выводится из уравнения гармонических колебаний пружинного маятника (4)).\newline

6. \textit{Какими параметрами характеризуются затухающие колебания?}

Уравнение движения: \[m\ddot{x}=-kx-hx,\eqno(5)\]
где \(h\) - коэффициент трения.

Решение этого уравнения: \[x=Ae^{-\delta t}\sin(\omega_1 t + \phi),\]
Параметры:

- коэффициент затухания \(\delta=\frac{h}{2m}\),

- частота \(\omega_1=\sqrt{\omega_0^2 - \delta^2}=\sqrt{\frac{k}{m}-\frac{h^2}{4m^2}}\),

- собственная частота \(\omega_0=\sqrt{\frac{k}{m}}\).

- амплитуда представляет собой убывающую по экспоненциальному закону функцию \(Ae^{-\delta t}\), где \(A\) - амплитуда начального колебания

- колебания не периодические, но выделяют условный период: \(T_1=\frac{2\pi}{\omega_1}\)

- время релаксации \(\tau=\frac{1}{\delta}\), за которое амплитуда уменьшается в \(e\) раз

- логарифмический декремент затухания: \(\theta =\frac{T_1}{t}\ln \frac{A_0}{A_t}=\delta T_1=\frac{T_1}{\tau}=\frac{1}{N_e}\), где \(N_e\) - число колебаний, совершённых за время \(t=\tau\)

- добротность \(Q=\frac{\pi}{\theta}=\pi N_e=\frac{\pi}{\delta T_1}\); для малых колебаний \(T_1 \approx T_0 \Rightarrow Q=\frac{\pi}{\delta T_0}, T_0 = \frac{2\pi}{\omega_0}\)

\end{document}

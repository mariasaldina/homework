\documentclass[12pt, letterpaper]{article}
\usepackage{makecell}
\usepackage[left=2cm,right=2cm,
    top=0.8cm,bottom=1.5cm,bindingoffset=0cm]{geometry}
\usepackage{amsmath,amsthm,amssymb}
\usepackage[utf8]{inputenc}
\usepackage[T1]{fontenc}
\usepackage[english,russian]{babel}
\usepackage{pgfplots}
\pgfplotsset{height=13cm,width=13cm,compat=1.9}
\usepackage{graphicx}

\title{Лабораторная работа №2}
\author{Выполнили: Политучая Инна и Салдина Мария}
\date{}

\begin{document}

\maketitle
\textbf{Наименование работы:} Измерение скорости полёта пули

\textbf{Цель работы:} ознакомление с баллистическим методом измерения, определение скорости полёта пули с помощью баллистического маятника, оценка точности метода измерения.

\textbf{Принадлежности:} баллистический маятник, пружинная пушка, шкала для отсчёта, шомпол, набор пуль, технические весы, набор гирь и разновесок.

\textbf{Рабочая формула:}  Скорость полёта пули рассчитывается по следующей формуле:
\[v = \frac{M+m}{m}S_0\sqrt{\frac{g}{l}},\eqno(1)\]
где \(M\) - масса маятника, в г,

\(m\) - масса пули, в г,

\(g\) - ускорение свободного падения, \(g=9800\) мм/с\(^2\)

\(S_0\) - расстояние, на которое отклоняется пуля, мм

\(l\) - расстояние от оси вращения до цента тяжести маятника, в мм.

\begin{center}
\Large{\textbf{Ход работы}}\
\end{center}

Масса маятника \(M = 2,178 \pm 0,001\) кг, длина нити \(l = 2,88 \pm 0,01\) м.

Погрешность измерения массы пуль \(\Delta m = 0,1\) г.

\begin{table}[h]
\caption{\label{tab:bolts} Измеренные и рассчитанные величины}
\begin{center}
 \begin{tabular}{|c|c|c|c|c|c|c|c|c|} 
 \hline
\thead{\\Номер\\опыта} & \thead{\\Номер\\пули}  & \(m\), г & \(S_0\), мм & \(\overline{S_0}\), мм & \(v\), м/с & \(\overline{v}\), м/с & \(|\Delta v|\), м/с & \(|\overline{\Delta v}|\), м/с \\ [0.5ex] 
 \hline
 1 & 1 & 9 & 11 & 11,6 & 4930,78 & 5199,73 & 268,95 & 215,16 \\ 
 \hline
 2 & 1 & 9 & 12 & 11,6 & 5379,03 & 5199,73 & 179,3 & 215,16 \\
 \hline
 3 & 1 & 9 & 11 & 11,6 & 4930,78 & 5199,73 & 268,95 & 215,16 \\
 \hline
 4 & 1 & 9 & 12 & 11,6 & 5379,03 & 5199,73 & 179,3 & 215,16 \\ 
 \hline
 5 & 1 & 9 & 12 & 11,6 & 5379,03 & 5199,73 & 179,3 & 215,16 \\
 \hline
 6 & 2 & 6,7 & 8 & 8,4 & 4811,98 & 5052,58 & 240,6 & 288,72 \\
 \hline
 7 & 2 & 6,7 & 8 & 8,4 & 4811,98 & 5052,58 & 240,6 & 288,72 \\ 
 \hline
 8 & 2 & 6,7 & 8 & 8,4 & 4811,98 & 5052,58 & 240,6 & 288,72  \\
 \hline
 9 & 2 & 6,7 & 9 & 8,4 & 5413,48 & 5052,58 & 360,9 & 288,72 \\
 \hline
 10 & 2 & 6,7 & 9 & 8,4 & 5413,48 & 5052,58 & 360,9 & 288,72 \\
 \hline
 11 & 3 & 3,4 & 5 & 4,4 & 5917,567 & 5207,46 & 710,107 & 568,09 \\
 \hline
 12 & 3 & 3,4 & 4 & 4,4 & 4734,05 & 5207,46 & 473,41 & 568,09 \\
 \hline
 13 & 3 & 3,4 & 5 & 4,4 & 5917,567 & 5207,46 & 710,107 & 568,09 \\
 \hline
 14 & 3 & 3,4 & 4 & 4,4 & 4734,05 & 5207,46 & 473,41 & 568,09 \\
 \hline
 15 & 3 & 3,4 & 4 & 4,4 & 4734,05 & 5207,46 & 473,41 & 568,09 \\
 \hline
\end{tabular}
\end{center}
\end{table}

Рассчитаем максимальную относительную погрешность метода измерений (см. таблицу 2: Результаты):

\[\frac{\Delta v}{v}=\frac{\Delta M}{M}+\frac{\Delta m}{m}+\frac{\Delta g}{2g}+\frac{\Delta l}{2l}+\frac{\Delta S_0}{S_0}\]

В качестве погрешностей измерений подставим погрешности отсчитывания средств измерений.

Для первой пули: \(\frac{\Delta v}{v}=\frac{0,001}{2,178}+\frac{0,1}{9}+\frac{0,1}{2 \cdot 9,8}+\frac{0,01}{2 \cdot 2,88}+\frac{1}{11,6}=10,46\%\)\newline
Для второй: \(\frac{\Delta v}{v}=\frac{0,001}{2,178}+\frac{0,1}{6,7}+\frac{0,1}{2 \cdot 9,8}+\frac{0,01}{2 \cdot 2,88}+\frac{1}{8,4}=14,13\%\)\newline
Для третьей: \(\frac{\Delta v}{v}=\frac{0,001}{2,178}+\frac{0,1}{3,4}+\frac{0,1}{2 \cdot 9,8}+\frac{0,01}{2 \cdot 2,88}+\frac{1}{8,4}=15,58\%\)

Окончательный результат: \(v=\overline{v}\pm\overline{\Delta v}\)

Относительная погрешность: \(\delta_V=\pm \frac{\overline{\Delta v}}{\overline{v}}\cdot 100\%\)

\textbf{Вывод:}\ мы ознакомились с баллистическим методом измерения.

Определили скорости полёта трёх пуль с помощью баллистического маятника, провели оценку точности метода измерения. Погрешность, полученная экспериментально, не превышает максимальной относительной погрешности метода измерений. Это говорит о том, что скорости пуль в рамках данного метода измерены достаточно точно.

\begin{table}[h]
\caption{\label{tab:bolts} Результаты}
\begin{center}
 \begin{tabular}{|c|c|c|c|} 
 \hline
\thead{\\Номер\\пули} & \thead{\\Скорость} & \thead{\\Относительная\\погрешность\\результата\\измерений} & \thead{\\Максимальная\\относительная\\погрешность\\метода\\измерений} \\ [0.5ex] 
 \hline
 1 & 5199,73 \(\pm\) 215,16 & 4,14\% & 10,46\% \\ 
 \hline
 2 & 5052,58 \(\pm\) 288,72 & 5,71\% & 14,13\% \\
 \hline
 3 & 5207,46 \(\pm\) 568,09 & 10,9\% & 15,58\% \\
 \hline
\end{tabular}
\end{center}
\end{table}

\begin{center}
\Large{\textbf{\newline\newline\newline Контрольные вопросы}}\
\end{center}

1. \textit{В чём заключается баллистический метод измерения скорости полёта пули?}

Баллистический маятник - прибор, применяемый для измерения начальных скоростей пуль или снарядов. Его выполняют в таком виде, чтобы его можно было рассматривать как математический маятник.

Метод баллистического маятника сводит измерение скорости пули к измерению отклонения сравнительно медленно движущегося маятника после абсолютно неупругого удара с пулей.

Нужно измерить массы пуль и съёмного внутреннего цилиндра маятника, установить ось маятника горизонтально по направлению ствола пушки и шкалу параллельно оси маятника вблизи его визира. Затем произвести выстрел и снять отсчёт смещения маятника.\newline

2. \textit{При каких условиях баллистический маятник можно принять за математический?}

Математический маятник - мат. точка, подвешенная на нерастяжимой невесомой нити и колеблющаяся под действием силы тяжести.

Ответ: если размерами цилиндра по сравнению с длиной нити можно пренебречь; если можно пренебречь растяжением и массой нити, сопротивлением воздуха. \newline

3. \textit{Сформулируйте законы сохранения импульса и механической энергии и укажите, как они используются при выводе рабочей формулы.}

Закон сохранения импульса: сумма импульсов всех тел системы есть величина постоянная (по величине и направлению), если векторная сумма внешних сил, действующих на систему тел, равна нулю.

Закон сохранения механической энергии: полная механическая энергия мат. точки (тела, частицы) в потенциальном поле (в консервативной системе) постоянна.

Система наз. консервативной, если она находится под действием только консервативных сил.

Закон сохранения импульса:
\[m\overline{v}=(M+m)\overline{V}\]
где \(\overline{v}\) - вектор скорости пули до удара,

\(\overline{V}\) - вектор сокрости пули сразу после удара

Для выполнения закона сохранения импульса необходимо, чтобы сумма внешних сил для замкнутой системы была равна нулю. В данном случае внешние силы: сила тяжести, сила натяжения нитей, мгновенная ударная сила (возникает в точке подвеса маятника во время удара). В момент удара система незамкнутая.

Закон сохранения импульса выполняется во время удара при условии, что:

1. вектор скорости пули направлен по прямой, проходящей через центр тяжести маятника (в данном случае центр качания = центру тяжести)

2. вектор \(\overline{v}\) направлен перпендикулярно плоскости, в которой лежат ось качания \(AA'\) и точка центра тяжести покоящегося маятника, т.е. в направлении оси Ox

3. продолжительность импульса должна быть настолько малой, чтобы маятник к концу удара не успевал существенно отклониться от положения равновесия (длинная нить + высокая вязкость вещества в маятнике)

Тогда: \(v=\frac{M+m}{m}V\)

Закон сохранения энергии:
\[\frac{(M+m)V^2}{2}=(M+m)gh\]
Тогда \(V=\sqrt{2gh}\),

где h - наибольшая высота подъёма центра тяжести маятника с пулей.

\[h=l-l\cos \alpha =2l\sin^2 \frac{\alpha}{2}\],

где \(\alpha\) - максимальный угол отклонения; т.к. он мал, то
\[\sin \alpha = \frac{S_0}{l} \approx \alpha \approx 2 \sin \frac{\alpha}{2}\]

тогда \(V=2\sqrt{gl} \sin \frac{\alpha}{2} = S_0 \sqrt{\frac{g}{l}}\)\newline

4. \textit{Как определяется максимальная относительная погрешность метода измерений?}

Складываются относительные погрешности всех переменных, входящих в формулу, умноженные на показатель их степени.

В качестве погрешностей измерений подставляют погрешности отсчитывания средств измерений.

\[\frac{\Delta v}{v}=\frac{\Delta M}{M}+\frac{\Delta m}{m}+\frac{\Delta g}{2g}+\frac{\Delta l}{2l}+\frac{\Delta S_0}{S_0}\].\newline

5. \textit{Какими факторами ограничивается точность измерения скорости полёта пули в опыте?}

Точность измерения ограничивается тем, что реальная система в момент удара является незамкнутой. На систему маятник-пуля действуют сила тяжести, сила натяжения нитей, мгновенная ударная сила. Кроме того, мы пренебрегли силой сопротивления воздуха. Во время удара и после него внешние силы не скомпенсированы.

\end{document}
